\documentclass{article}
\usepackage[cm]{fullpage} %very small margins (around 1.5cm)
\usepackage[utf8]{inputenc} %sets input encoding to UTF-8, needed for Polish, Japanese, etc.
\usepackage[T1]{fontenc} %needed for Polish characters
\usepackage{lmodern} %this font handles Polish characters properly

\PassOptionsToPackage{usenames,dvipsnames,svgnames}{xcolor}

%this file must be included in the preamble

\usepackage{graphicx} %use \includegraphics[scale=1.00]{file.jpg} for images
\usepackage[unicode]{hyperref}

\hypersetup{
  bookmarksopen,
  bookmarksopenlevel=1,
  pdfborder={0 0 0},
  %pdfpagelayout={SinglePage},
  %pdfstartview={FitH},
  pdfdisplaydoctitle
}

\def\logoleft#1{\gdef\logoleftpath{#1}}
\def\logoleftpath{\errmessage{Please define the logoleft}}

\def\logoleftscale#1{\gdef\logoleftscaletext{#1}}
\def\logoleftscaletext{\errmessage{Please define the logoleftscale}}

\def\logoright#1{\gdef\logorightpath{#1}}
\def\logorightpath{\errmessage{Please define the logoright}}

\def\logorightscale#1{\gdef\logorightscaletext{#1}}
\def\logorightscaletext{\errmessage{Please define the logorightscale}}

\def\title#1{\gdef\titletext{#1}\hypersetup{pdftitle={#1}}}
\def\titletext{\errmessage{Please define the title}}

\def\author#1{\gdef\authortext{#1}\hypersetup{pdfauthor={#1}}}
\def\authortext{\errmessage{Please define the author}}

\def\supervisor#1{\gdef\supervisortext{#1}}
\def\supervisortext{\errmessage{Please define the supervisor}}

\def\date#1{\gdef\datetext{#1}}
\def\datetext{\errmessage{Please define the date}}

\def\university#1{\gdef\universitytext{#1}}
\def\universitytext{\errmessage{Please define the university}}

\def\faculty#1{\gdef\facultytext{#1}}
\def\facultytext{\errmessage{Please define the faculty}}

\def\location#1{\gdef\locationtext{#1}}
\def\locationtext{\errmessage{Please define the location}}

\def\description#1{\gdef\descriptiontext{#1}}
\def\descriptiontext{\errmessage{Please define the description}}

\renewcommand{\maketitle}{
\pdfbookmark[1]{Title page}{titlepage}
\begin{titlepage}
\begin{center}
  \begin{minipage}{.15\textwidth}
    \begin{flushleft}
      \includegraphics[scale=\logoleftscaletext]{\logoleftpath}
    \end{flushleft}
  \end{minipage}
  \begin{minipage}{.66\textwidth}
    \begin{center}
      { \Large \universitytext }
    \end{center}
    
    \begin{center}
      { \Large \facultytext }
    \end{center}
  \end{minipage}
  \begin{minipage}{.15\textwidth}
    \begin{flushright}
      \includegraphics[scale=\logorightscaletext]{\logorightpath}
    \end{flushright}
  \end{minipage}
\end{center}

\vspace*{\fill}
  
\begin{center}
  \textsc{\Huge \titletext }
\end{center}
  
\begin{center}
  \textsc{ \descriptiontext }
\end{center}

\vspace*{\fill}

\hspace{.5\textwidth}
\begin{minipage}[t]{.4\textwidth}
  \begin{flushleft}
    {\large Author: \newline \authortext }
  \end{flushleft}
  
  \begin{flushleft}
    {\large Supervisor: \newline \supervisortext }
  \end{flushleft}
\end{minipage}

\vspace{50pt}

\begin{center}
  {\large\locationtext, \datetext}
\end{center}
\end{titlepage}
}


\logoleft{../../graphics/logo_pw.jpg}
\logoleftscale{0.395}
\logoright{../../graphics/logo_mini.png}
\logorightscale{0.16}
\university{Warsaw University of Technology}
\faculty{Faculty of Mathematics and~Computer~Science}
\location{Warsaw}
\supervisor{dr Lucjan Stapp}
\title{Testing of ReCon}
\description{an application for building finite-state machine that is equivalent to a given regular
expression \newline and for simulating machine's evaluation of a given word}
\author{Mateusz Bysiek}
\date{6 Jun 2013}

%this file must be included in the preamble

\usepackage{booktabs} %use \begin{tabular}{} with \toprule \midrule \bottomrule
\usepackage{tabulary}

\def\company#1{\gdef\companytext{#1}}
\def\companytext{\errmessage{Please define the company}}

\def\documentsubject#1{\gdef\documentsubjecttext{#1}}
\def\documentsubjecttext{\errmessage{Please define the documentsubject}}

\def\documenttopicslist#1{\gdef\documenttopicslisttext{#1}}
\def\documenttopicslisttext{\errmessage{Please define the documenttopicslist}}

\def\filename#1{\gdef\filenametext{#1}}
\def\filenametext{\errmessage{Please define the filename}}

\def\version#1{\gdef\versiontext{#1}}
\def\versiontext{\errmessage{Please define the version}}

\def\status#1{\gdef\statustext{#1}}
\def\statustext{\errmessage{Please define the status}}

\def\openingdate#1{\gdef\openingdatetext{#1}}
\def\openingdatetext{\errmessage{Please define the openingdate}}

\def\documentsummary#1{\gdef\documentsummarytext{#1}}
\def\documentsummarytext{\errmessage{Please define the documentsummary}}

% \def\#1{\gdef\text{#1}}
% \def\text{\errmessage{Please define the }}

\newcommand{\tableheader}[1]{\textbf{\large {#1}}}
\newcommand{\fieldheader}[1]{\textbf{\footnotesize {#1}}}

\newcommand{\documentmetric}{
\pdfbookmark[1]{Document metric}{docmetric}
\begin{center}
\begin{tabulary}{\textwidth}{LJLJLJ}
\toprule
\multicolumn{6}{l}{\tableheader{Document metric}} \\
\midrule
\fieldheader{Project} & \multicolumn{3}{l}{\textsc{\titletext}} 
& \fieldheader{Company} & \companytext \\
\midrule    
\fieldheader{Document name} & \multicolumn{5}{l}{\documentsubjecttext} \\
\midrule
\fieldheader{Document topics} & \multicolumn{5}{l}{\documenttopicslisttext} \\
\midrule
\fieldheader{Author} & \multicolumn{5}{l}{\authortext} \\
\midrule
\fieldheader{File} & \multicolumn{5}{l}{\filenametext} \\
\midrule
\fieldheader{Version no.} & \versiontext 
& \fieldheader{Status} & \statustext
& \fieldheader{Opening date} & \openingdatetext \\
\midrule
\fieldheader{Summary} & \multicolumn{5}{l}{\parbox{.80\textwidth}{\documentsummarytext}
} \\
\midrule
\fieldheader{Authorized by} & \multicolumn{3}{l}{\parbox{.40\textwidth}{\supervisortext}} 
& \fieldheader{Last modification date} & \datetext \\
\bottomrule
\multicolumn{6}{l}{\makebox[.96\textwidth]{}} \\
\end{tabulary}
\end{center}
}

\newcommand{\historyofchanges}[1]{
\pdfbookmark[1]{History of changes}{dochistory}
\begin{center}
\begin{tabulary}{\textwidth}{LLLJ}
\toprule
\multicolumn{4}{l}{\tableheader{History of changes}} \\
\midrule
\fieldheader{Version} & \fieldheader{Date} & \fieldheader{Author} & \fieldheader{Description} \hspace*{\fill} \\
#1
\bottomrule
\multicolumn{4}{l}{\makebox[.96\textwidth]{}} \\
\end{tabulary}
\end{center}
}

\newcommand{\change}[4]{
\midrule
{#1} & {#2} & {#3} & {#4} \\
}


\company{WUT}
\documentsubject{test report}
\documenttopicslist{conformance tests, testing scenarios, detected anomailes, general recommendations}

\documentsummary{Author provides definitions of test scenarios that include expected application
behavior, and performs them, writnig down all results, and he compares the contents
of arfifacts from previous stages of the project to the actual application experience.}

\openingdate{1 Jun 2013}
\version{1.0}
\status{final}
\filename{bysiekm-testing.pdf}

%this file must be included in the preamble

\usepackage{color}
\usepackage[usenames,dvipsnames]{xcolor}
\usepackage{lastpage} %page x of y with: \cfoot{\thepage{} of \pageref{LastPage}}
\usepackage{fancyhdr} % also needed for footers

\fancyhead{}
\fancyfoot{}
\lfoot{{ \footnotesize \textcolor{gray}{\authortext} }}
\cfoot{{ \footnotesize page \thepage{} of \pageref{LastPage} }}
\rfoot{\textsc{ \footnotesize \textcolor{gray}{\titletext, v.\versiontext} }}
\pagestyle{fancy}
\renewcommand{\headrulewidth}{0pt}
\renewcommand{\footrulewidth}{0pt}


\usepackage{amsmath}
\usepackage{amsfonts}
\usepackage{multicol} %use \begin{multicols}{#} for # columns
\usepackage{enumitem} %remove vertical space in itemize with: [noitemsep,nolistsep]
\usepackage{tikz}
%\usepackage{minted} %use \begin{minted}[mathescape,linenos,numbersep=5pt,gobble=0,framesep=2mm]{c++}
\usepackage{tabularx}
\usepackage{../../includes/pgf-umlcd}

\usetikzlibrary{arrows,positioning,automata}

\newcommand{\writehere}{\textbf{\textcolor{red}{Write here!}}}

\newcommand{\errorminor}{Minor, }

\newcommand{\errormajor}{\textbf{Major}, }

\newcommand{\errorcritical}{\textbf{\textcolor{red}{Critical}}, }

\begin{document}

\maketitle

\newpage

\setcounter{page}{2}

\documentmetric

\vspace*{\fill}

\historyofchanges{
\change{0.01}{\openingdatetext}{\authortext}{creation of template for the document}
\change{0.02}{23 Feb 2013}{\authortext}{outline of the features section}
\change{0.03}{25 Feb 2013}{\authortext}{added definitions section}
\change{0.04}{25 Feb 2013}{\authortext}{added algorithm section}
\change{0.05}{27 Feb 2013}{\authortext}{added GUI mockup, moved definitions}
\change{0.06}{28 Feb 2013}{\authortext}{added use cases}
\change{0.07}{8 Mar 2013}{\authortext}{removed GUI mockup}
%\change{0.08}{8 Mar 2013}{\authortext}{removed GUI mockup}
\change{\versiontext}{10 Mar 2013}{\authortext}{added word evaluation feature}
%\change{\versiontext}{\datetext}{\authortext}{added use cases}
}

\newpage

\tableofcontents

\newpage


\section*{Abstract}

This is a acceptance test report for an application for building finite-state machine that is
equivalent to a given regular expression \newline and for simulating machine's evaluation of a given
word. Application's name is ReCon.

Author of the document lists test scenarios that are relevant for working
application that is meant to solve a well defined language\mbox{-}theory related problem, he does it
from end-user perspective, i.e. provides definitions of test scenarios that include expected
application behavior, and performs them, writnig down all results.

He also compares the contents of arfifacts from previous stages of the project to the actual
application experience, he does it from end-user perspective, adversary perspective, and technical
advisor perspective.

\section{Release package}
Release package contains application executable, together with business
analysis and technical analysis. The application executable is accompanied by several supporting
files.

\subsection{Anomalies}
There are some obsolete files in the release package:
\begin{itemize}%[noitemsep,nolistsep]
  \item *.pdb
  \item RegexFiniteAutomaton.vshost.*
\end{itemize}

The presence of additional executable may confuse some users. Presence of debugging symbols may
help the adversary to reverse engineer the application, which may be undesired.

Otherwise, release package is well prepared - it contains all required files and the application can
be launched without the need for any initial environment configuration.

\section{Conformance with the documentation}

\subsection{Business analysis}
The application fulfills most of the requirements defined in the business analysis. However there
are some parts of the application that are in contradiction with the business analysis, and there
was no indication in technical analysis that the requirements with regard these parts have changed.

\subsection{Business analysis - nonconforming parts}

\subsubsection{Section 2.1.2}
Per ``Upper index characters should be preceded by a “\^{}” sign.'' the language for regular
expressions used in the application is partially wrong. It is important to remark that in most part
in conforms with the business analysis.

\subsubsection{Section 2.2.3}
Per ``user of the application should be able to observe computation for a given regular expression
on the generated automata'' and ``The transition is made when the user clicks Next step button'' the
application is missing step-by-step machine construction.

\newpage

\subsection{Technical analysis}
When main features are concerned, the application realizes the main purposes described in the
documentation, however there are some details that require attention.

\subsection{Technical analysis - nonconforming parts}

\subsubsection{Section ``General Requirements''}
Per ``All states are marked as black ellipses, all transitions as directed nodes (also black).'' the
coloring scheme used in the application is wrong.

\subsubsection{Section ``GUI''}
Per ``in the input section there are two radio button’s groups. One for entering the input and the
second one for selecting computation mode.'' the application has two radio buttton groups, but one
of these groups consists of only one radio option.

\section{Startup}
There are no general preconditions for these tests. 

\subsection{Testing scenarios}

\begin{center}
\begin{tabularx}{\textwidth}{llXX}
\toprule
\textbf{Nr} & \textbf{Scenario} & \textbf{Input} & \textbf{Expected Result}  \\
\midrule
1 & starting app
& double-click on app executable
& app opens \\
\bottomrule
\end{tabularx}
\end{center}

\subsection{Anomalies}
None detected.

\newpage

\section{Regular expression handling}
Before theses thests: launch application, and wait for 2 seconds.

\subsection{Testing scenarios}

\begin{center}
\begin{tabularx}{\textwidth}{llXX}
\toprule
\textbf{Nr} & \textbf{Scenario} & \textbf{Input} & \textbf{Expected Result}  \\
\midrule
1 & empty text box
& leave text box empty, click convert
& error indication is seen \\
\midrule
2 & empty word
& enter ``$ \$ $'' in the text box, click convert
& computation ends normally, automaton is seen \\
\midrule
3 & letter ``a''
& enter ``$ a $'' in the text box, click convert
& computation ends normally, automaton is seen \\
\midrule
4 & nonsense input
& enter ``$ a^*(^* $'' in the text box, click convert
& error indication is seen \\
\midrule
5 & advanced test 1
& enter ``$ (a+b)(a+b)(a+b)(a+b) $'' \newline in the text box, click convert
& computation ends normally, automaton is seen \\
\midrule
6 & advanced test 2
& enter ``$ (a+b)^* $'' in the text box, click convert
& computation ends normally, automaton is seen \\
\midrule
7 & kleene star
& enter ``$ a^* $'' in the text box, click convert
& computation ends normally, automaton is seen \\
\midrule
8 & kleene plus
& enter ``$a$\^{}'' in the text box, click convert
& computation ends normally, automaton is seen \\
\midrule
9 & special characters
& enter ``\#'', ``.'', ``\_'' or ``-'' in the text box, click convert
& error indication is seen \\
\bottomrule
\end{tabularx}
\end{center}

\subsection{Anomalies}
Detected in the following scenarios:
\begin{itemize}

  \item 1: \errorcritical application stops working and exits.

  \item 2: \errorcritical application stops working and exits.

  \item 4: \errorcritical application stops working and exits.

  \item 5, 6 and 7: \errorminor resulting automatons are not minimal.

  \item 9: \errorcritical application stops working and exits.

\end{itemize}

\subsection{Comments}
\errormajor I have not found any method of launching conversion of regular expression into
finite-state machine in step-by-step mode - the only choice is to do immediate computation.

\errormajor there is an issue of missing step-by-stem finite-state machine construction feature.
Either the module is missing, or a user control that enables the user to choose construction mode
was forgotten.

\newpage

\section{Word handling}
Before these tests: launch application, enter ``$ a^* $'' in the regexp text box, click convert.

\subsection{Testing scenarios}

\begin{center}
\begin{tabularx}{\textwidth}{llXX}
\toprule
\textbf{Nr} & \textbf{Scenario} & \textbf{Input} & \textbf{Expected Result}  \\
\midrule
1 & accepted word
& select normal mode, enter ``aaa'', click observe
& computation ends normally, information that word was accepted is seen \\
\midrule
2 & rejected word
& select normal mode, enter ``aab'', click observe
& computation ends normally, information that word was rejected is seen \\
\midrule
3 & accepted word, stepping
& select step-by-step mode, enter ``aaa'', click observe
& computation ends normally, information that word was accepted is seen \\
\midrule
4 & rejected word, stepping
& select step-by-step mode, enter ``aab'', click observe
& computation ends normally, information that word was rejected is seen \\
\midrule
5 & accepted empty word
& select normal mode, enter nothing, click observe
& computation ends normally, information that word was accepted is seen \\
\midrule
6 & accepted empty word, stepping
& select step-by-step mode, enter nothing, click observe
& computation ends normally, information that word was accepted is seen \\
\bottomrule
\end{tabularx}
\end{center}

\subsection{Anomalies}
Detected in the following scenarios:
\begin{itemize}

  \item 2 and 4: \errorcritical application stops working and exits.

\end{itemize}

\section{Closing}
Before these tests: launch application.

\subsection{Testing scenarios}

\begin{center}
\begin{tabularx}{\textwidth}{llXX}
\toprule
\textbf{Nr} & \textbf{Scenario} & \textbf{Input} & \textbf{Expected Result}  \\
\midrule
1 & exiting while app idle
& wait for 2 seconds, click on X in app window
& application shuts down \\
\midrule
2 & exiting while app computing
& enter ``$ a^{128} $'' (128 ``a''s) in the text box, click convert, immediately click on X in app window
& application shuts down \\
\bottomrule
\end{tabularx}
\end{center}

\subsection{Anomalies}
Detected in the following scenarios:
\begin{itemize}

  \item 2: \errormajor the X button is not responding until the application finishes computation

\end{itemize}

\newpage

\section{Summary}
Application performs well for a very specific input data. It handles typical regular expressions,
and produces not always minimal, but correct finite-state machines. However, application is unable
to handle any unexpected content, therefore in my opinion error handling is either not present or it
is not properly done.

In most part, application conforms with requirements defined in the previous stages of the project,
but there are some features that are either required but not implemented, or implemented but in a
way that is not correct from the formal requirements point of view. 

There is also an issue of UI blocking that is an application wide issue - whenever application is
occupied, user cannot perform any other tasks, but there is no indication of this, the application
window simply hangs until the application becomes idle.

In conclusion, the ReCon application is conditionally accepted as it is. There is one condition of
acceptance: it is that error handling must be done properly, so that no unexpected application
shutdown happens when the tests defined above are performed.

\end{document}
